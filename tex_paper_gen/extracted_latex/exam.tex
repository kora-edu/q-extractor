
\documentclass[12pt,addpoints]{exam}
\usepackage{fontspec}
\usepackage{unicode-math}
\usepackage{amsmath}
\usepackage[a4paper,margin=1in]{geometry}
\usepackage{setspace}


\begin{document}
\title{json -> latex convert test}
\author{kora}
\date{}
\maketitle
\section*{Questions}
\begin{questions}
\pointsinrightmargin
\bracketedpoints
\question[5] Write \(\frac{4}{2 - \sqrt{5}}\) in the form \(a + b\sqrt{5}\), where \(a\) and \(b\) are rational numbers.
\fillwithlines{3cm}
\question[5] One root of the equation \( z^3 - 3z + p = 0 \) is \( z = 2 - 3i \). If \( p \) is a real number, find the value of \( p \) and the other roots of the equation.
\fillwithlines{3cm}
\question[5] If \( z = 1 + i \) and \( w = \frac{1}{z} + i \), find the exact value of \(\arg(w)\).
\fillwithlines{3cm}
\question[5] If \( u = 6 + ki \) and \( v = 4 + ki \), find \( k \) if \(\arg(u \cdot v) = \frac{\pi}{4}\).
\fillwithlines{3cm}
\question[5] What is the remainder when \( x^3 + 4x^2 + 3x - 9 \) is divided by \( x + 2 \)?
\fillwithlines{3cm}
\question[5] If \( u = 2 \text{cis} \frac{2\pi}{3} \) and \( v = 6 \text{cis} \frac{\pi}{2} \), write \( \frac{u}{v} \) in polar form.
\fillwithlines{3cm}
\question[5] Find the equation whose roots are three times those of \( x^2 + 9x - 12 = 0 \).
\fillwithlines{3cm}
\question[5] Given \( u = x + iy \) and the equation \( au^2 + bu + c = 0 \), prove that if \( u \) is a solution, then its complex conjugate \( \overline{u} \) is also a solution, i.e., \( a\overline{u}^2 + b\overline{u} + c = 0 \).
\fillwithlines{3cm}
\question[5] Describe fully the locus of the points representing \( z \) if \(\frac{z + 2i}{z - 2i}\) is purely imaginary.
\fillwithlines{3cm}
\question[5] Solve the equation \( z^2 + 6z + 20 = 0 \). Express the solutions in the form \( z = a + \sqrt{b}i \), where \( a \) and \( b \) are integers.
\fillwithlines{3cm}
\question[5] Given the complex numbers \( p = 3 + 4i \) and \( q = 2 - 3i \), find \( p\overline{q} \), expressing your answer in the rectangular form \( a + bi \).
\fillwithlines{3cm}
\question[5] Solve the following equation for \( x \) in terms of \( p \):

\[
\sqrt{x} - 3 = \sqrt{x - p}
\]
\fillwithlines{3cm}
\question[5] Find all the solutions of the equation \( z^3 + n = 0 \), where \( n \) is a positive real number. Write your solutions in polar form as expressions in terms of \( n \).
\fillwithlines{3cm}
\question[5] Solve the equation \(6 + x = 4\sqrt{3x + k}\) for \(x\), and determine the condition on \(k\) such that the equation has no real roots.
\fillwithlines{3cm}
\question[5] Given that \( x - 2 \) is a factor of \( g(x) = x^3 - 2px^2 + px - 5 \), find the value of \( p \) where \( p \) is real.
\fillwithlines{3cm}
\question[5] If \( u = 3 - 3i \), find \( u^4 \) in the form \( r \text{cis} \theta \).
\fillwithlines{3cm}
\question[5] Solve the equation \( z^4 = -4k^2i \), where \( k \) is a real number. Write your solutions in polar form in terms of \( k \).
\fillwithlines{3cm}
\question[5] Find the equation of the locus described by \( |z - 1 + 2i| = |z + 1| \).
\fillwithlines{3cm}
\question[5] Given that \( w = 2 \text{cis} \frac{\pi}{3} \), find \( w^4 \). Give your answer in the form \( a + bi \), where \( a \) and \( b \) are real numbers.
\fillwithlines{3cm}
\question[5] Given that \( w = 2 - 3i \) is a solution of the equation \( 3w^3 - 14w^2 + Aw - 26 = 0 \), where \( A \) is real, find the value of \( A \) and the other two solutions of the equation.
\fillwithlines{3cm}
\question[5] A complex number \( z \) satisfies \( |z - 3 - 4i| = 2 \). Sketch the locus of points that represents \( z \) on the Argand diagram.
\fillwithlines{3cm}
\question[5] The complex number \( z \) is given by \( z = \frac{1+3i}{p+qi} \), where \( p \) and \( q \) are real numbers and \( p > q > 0 \). Given that \(\text{Arg}(z) = \frac{\pi}{4}\), show that \( p - 2q = 0 \).
\fillwithlines{3cm}
\question[5] Expand and simplify as far as possible the following expression: \((2 - \sqrt{3})(5 + 2\sqrt{3})(4 - 3\sqrt{3})\). Give your answer in the form \( a + b\sqrt{3} \), where \( a \) and \( b \) are real numbers.
\fillwithlines{3cm}
\question[5] The complex numbers \( p \) and \( q \) are represented on the Argand diagram. If \( r = 2p - 3q \), find the value of \( r \) and mark it on the Argand diagram.
\fillwithlines{3cm}
\question[5] For what values of \( p \), where \( p \) is real, does the graph of \( y = px^2 - 4px + 1 \) not intersect the \( x \)-axis?
\fillwithlines{3cm}
\question[5] Given that \( z = 3 + 2i \), find the value of \( \overline{z}^2 + \frac{1}{z^2} \), giving your answer in the form \( a + bi \), where \( a \) and \( b \) are real.
\fillwithlines{3cm}
\question[5] Given that \( \alpha, \beta, \) and \( \gamma \) are the three roots of the cubic equation \( ax^3 + bx^2 + cx + d = 0 \), where \( a, b, c, \) and \( d \) are real numbers, prove the following relationships:

(i) \(\alpha + \beta + \gamma = -b/a\)

(ii) \(\alpha\beta + \beta\gamma + \alpha\gamma = c/a\)

(iii) \(\alpha\beta\gamma = -d/a\)
\fillwithlines{3cm}
\question[5] (ii) Hence prove that \(\alpha^2\beta\gamma + \alpha\beta^2\gamma + \alpha\beta\gamma^2 = \frac{bd}{a^2}\) given that \(\alpha\), \(\beta\), and \(\gamma\) are the roots of the cubic polynomial \(ax^3 + bx^2 + cx + d = 0\).
\fillwithlines{3cm}
\question[5] Solve the equation \(x^2 - 8x + 4 = 0\). Write your answer in the form \(a \pm b\sqrt{c}\), where \(a\), \(b\), and \(c\) are integers and \(b \neq 1\).
\fillwithlines{3cm}
\question[5] If \( u = 1 + \sqrt{3}i \), show \( u^3 \) on the Argand diagram.
\fillwithlines{3cm}
\question[5] Given the complex numbers \( v = 3 - 7i \) and \( w = -4 + 6i \), find the real numbers \( p \) and \( q \) such that \( pv + qw = 6.5 - 11i \).
\fillwithlines{3cm}
\question[5] Prove that the roots of the equation \( 3x^2 + (2c + 1)x - (c + 3) = 0 \) are always real for all values of \( c \), where \( c \) is real.
\fillwithlines{3cm}
\question[5] If the polynomials \(x^2 + bx + c\) and \(x^2 + dx + e\) have a common factor of \((x - p)\), prove that \((e - c) / (b - d) = p\), where \(b, c, d, e,\) and \(p\) are all real numbers.
\fillwithlines{3cm}
\question[5] What is the remainder when \(2x^3 + x^2 - 5x + 7\) is divided by \(x + 3\)?
\fillwithlines{3cm}
\question[5] Express the complex number \((2 + 3i) / (5 + i)\) in the form \(k(1 + i)\), where \(k\) is a real number. Find the value of \(k\).
\fillwithlines{3cm}
\question[5] Find real numbers \( A, B, \) and \( C \) such that

\[
\frac{1}{x^2(x-1)} = \frac{A}{x} + \frac{B}{x^2} + \frac{C}{x-1}
\]
\fillwithlines{3cm}
\question[5] Write the complex number \(\left( \frac{4i^7 - i}{1 + 2i} \right)^2\) in the form \( a + bi \), where \( a \) and \( b \) are real numbers.
\fillwithlines{3cm}
\question[5] Find the Cartesian equation of the locus described by \(\arg \left( \frac{z - 2}{z + 5} \right) = \frac{\pi}{4}\).
\fillwithlines{3cm}
\question[5] If \( z = 4 + 2i \) and \( w = -1 + 3i \), find \(\arg(zw)\).
\fillwithlines{3cm}
\question[5] For what real value(s) of \( k \) does the equation \( kx^2 + \frac{x}{k} + 2 = 0 \) have equal roots?
\fillwithlines{3cm}
\question[5] One solution of the equation \( 3w^3 + Aw^2 - 3w + 10 = 0 \) is \( w = -2 \). If \( A \) is a real number, find the value of \( A \) and the other two solutions of the equation.
\fillwithlines{3cm}
\question[5] Solve the equation \( z^3 = k + \sqrt{3} \, ki \), where \( k \) is real and positive. Write your solutions in polar form in terms of \( k \).
\fillwithlines{3cm}
\question[5] Find each of the roots of the equation \( z^5 - 1 = 0 \).
\fillwithlines{3cm}
\question[5] Let \( p \) be the root in part (i) with the smallest positive argument. Show that the roots in part (i) can be written as \( 1, p, p^2, p^3, p^4 \).

Parts:
i) Query: Find the fifth roots of unity.

ii) Query: Let \( p \) be the root with the smallest positive argument. Show that the roots can be written as \( 1, p, p^2, p^3, p^4 \).
\fillwithlines{3cm}
\question[5] Complex numbers \( p \) and \( q \) are represented on the Argand diagram. If \( s = p + q \), how do you determine the position of \( s \) on the Argand diagram?
\fillwithlines{3cm}
\question[5] Dividing \( 2x^3 + 5x^2 + Ax + 7 \) by \( x + 3 \) gives a remainder of 16. What is the value of \( A \)?
\fillwithlines{3cm}
\question[5] Solve the equation \( 5 - \sqrt{x} = \sqrt{x - p} \) for \( x \) in terms of \( p \).
\fillwithlines{3cm}
\question[5] If \( w = 1 + 2i \), find the value of \( w^2 + \frac{w}{\overline{w}} \), giving your answer in the form \( a + bi \), where \( a \) and \( b \) are real. You must clearly show each step of your working.
\fillwithlines{3cm}
\question[5] The locus described by \( |z - 2 + 3i| = |z - 1| \) is a straight line. Find the gradient of that line.
\fillwithlines{3cm}
\question[5] Solve the equation \(x^2 - 6x + 12 = 0\). Write your answer in the form \(a \pm \sqrt{b}i\), where \(a\) and \(b\) are rational numbers.
\fillwithlines{3cm}
\question[5] Given \( u = 2 + 3i \) and \( v = 5 + mi \), find the value of \( m \) if \( uv = 22 + 7i \).
\fillwithlines{3cm}
\question[5] Solve the equation \(z^3 = -8k^6\), where \(k\) is a real number. Write your solutions in polar form in terms of \(k\).
\fillwithlines{3cm}
\question[5] Prove that \(\left| \frac{4 + 2i}{1 + i} \right| = \sqrt{10}\).
\fillwithlines{3cm}
\question[5] Write \( \frac{5}{2+\sqrt{3}} \) in the form \( a + b\sqrt{c} \).
\fillwithlines{3cm}
\question[5] If \( v = 4 \text{cis} \frac{3\pi}{4} \) and \( w = 6 \text{cis} \frac{2\pi}{3} \), write the exact value of \( \frac{v}{w} \) in polar form.
\fillwithlines{3cm}
\question[5] Given that \( z = 3 - 4i \) is one solution of the equation \( z^3 - 8z^2 + Bz - 50 = 0 \), find the value of \( B \).
\fillwithlines{3cm}
\question[5] If \( u \) and \( v \) are complex numbers, prove that \( \overline{uv} = \overline{u} \cdot \overline{v} \).
\fillwithlines{3cm}
\question[5] Let \( u \) and \( v \) be two complex numbers such that \( |u+v|^2 = |u-v|^2 \). Prove that \( \overline{uv} \) is purely imaginary.
\fillwithlines{3cm}
\question[5] If \( u = 2 + 3i \) and \( v = 1 - 4i \), find \( \overline{u} - 3v \), giving your solution in the form \( a + bi \).
\fillwithlines{3cm}
\question[5] Write \( \frac{36}{5 - \sqrt{7}} \) in the form \( a + b\sqrt{7} \), where \( a \) and \( b \) are integers.
\fillwithlines{3cm}
\question[5] Given that one solution of the equation \( z^3 - 2z^2 + Bz - 30 = 0 \) is \( z = -2 - i \), and \( B \) is a real number, find the value of \( B \) and the other two solutions of the equation.
\fillwithlines{3cm}
\question[5] Find the Cartesian equation of the locus described by \( |z + 2 - 7i| = 2|z - 10 + 2i| \). Write your answer in the form \( (x + A)^2 + (y + B)^2 = K \).
\fillwithlines{3cm}
\question[5] Dividing \(x^3 - 2x^2 + 5x + d\) by \(x - 3\) gives a remainder of 13. Find the value of \(d\).
\fillwithlines{3cm}
\question[5] Simplify, as far as possible, the expression \(\sqrt{2k} \left( \sqrt{18k} - \sqrt{8k} \right)\).
\fillwithlines{3cm}
\question[5] Given that \(z\) and \(w\) are complex numbers such that \(z = -2 + 3i\) and \(zw = 15 - 3i\), find the exact value of \(\arg(w)\).
\fillwithlines{3cm}
\question[5] Solve the equation \( z^4 = \frac{m}{\sqrt{2}} + \frac{m}{\sqrt{2}} i \), where \( m \) is real and positive. Write your solutions in polar form in terms of \( m \).
\fillwithlines{3cm}
\question[5] Find all possible values of \( k \) that make \( u = \frac{k + 4i}{1 + ki} \) a purely real number.
\fillwithlines{3cm}
\question[5] If \( u = p^{3} \text{cis} \frac{\pi}{3} \) and \( v = p \text{cis} \frac{\pi}{8} \), write \( \frac{u}{v} \) in polar form.
\fillwithlines{3cm}
\question[5] Solve the equation \( x^2 - 6x + 14 = 0 \). Give your solution in the form \( a \pm \sqrt{b}i \), where \( a \) and \( b \) are rational numbers.
\fillwithlines{3cm}
\question[5] Given the expression \((3x^3 + 8x^2 - 2x + 11) / (x + 2) = 3x^2 + Ax + B + \frac{C}{x + 2}\), where A, B, and C are integers, find the values of A, B, and C.
\fillwithlines{3cm}
\question[5] Solve the equation \(\frac{8 + x}{x} = \sqrt{3}\), writing your solution in the form \(x = a + b\sqrt{3}\).
\fillwithlines{3cm}
\question[5] What is the remainder when \(2x^3 - 3x^2 + 4x + 3\) is divided by \(x - 2\)?
\fillwithlines{3cm}
\question[5] If \( u = m \text{cis} \frac{\pi}{3} \) and \( v = m^3 \text{cis} \frac{2\pi}{5} \), find \( uv \) in polar form.
\fillwithlines{3cm}
\question[5] Solve the equation \(2 + \sqrt{x} = \sqrt{x + k}\) for \(x\) in terms of \(k\).
\fillwithlines{3cm}
\question[5] Find the exact value(s) of \( k \) for which the equation \( k(1 + x^2) = 3 - 8x - x^2 \) has one repeated solution. Give your solution in the form \( k = a \pm \sqrt{b} \).
\fillwithlines{3cm}
\question[5] If \( z = a + bi \) and \( \frac{z}{\overline{z}} = c + di \), prove that \( c^2 + d^2 = 1 \).
\fillwithlines{3cm}
\question[5] Complex numbers \( u \) and \( v \) are represented on the Argand diagram. If \( w = u + \overline{v} \), how can \( w \) be shown on the Argand diagram?
\fillwithlines{3cm}
\question[5] Write \( \frac{6}{3 - \sqrt{7}} \) in the form \( a + b\sqrt{7} \).
\fillwithlines{3cm}
\question[5] One solution of the equation \( z^3 + Az^2 + 34z - 40 = 0 \) is \( z = 3 + i \). If \( A \) is a real number, find the value of \( A \) and the other two solutions of the equation.
\fillwithlines{3cm}
\question[5] If \( z = \frac{15}{1 - 2i} - 2i \), find \(\text{mod}(z)\). You must show all algebraic working.
\fillwithlines{3cm}
\question[5] The complex number \( u = 3 + mi \) is on the locus of points defined by \( |z - 8| = |z - 4 + 2i| \). Find the value of \( m \).
\fillwithlines{3cm}
\question[5] Given the complex numbers \( u = 3 - 2i \) and \( v = 2 + bi \), find the value of \( b \) if the product \( uv = 14 + 8i \).
\fillwithlines{3cm}
\question[5] Solve the equation \( x^2 - 6px + 4p^2 = 0 \) for \( x \) in terms of \( p \), expressing the solution in its simplest form.
\fillwithlines{3cm}
\question[5] Find the complex number \( w \), in the form \( x + iy \), if \(\arg(w) = \frac{\pi}{4}\) and \(|w \cdot \overline{w}| = 20\).
\fillwithlines{3cm}
\question[5] Solve the equation \(x^2 - 4x + 7 = 0\). Give your solution in the form \(a \pm \sqrt{b}i\), where \(a\) and \(b\) are rational numbers.
\fillwithlines{3cm}
\question[5] When the polynomial \(2x^3 - x^2 - 4x + p\) is divided by \(x - 3\), the remainder is 38. Find the value of \(p\).
\fillwithlines{3cm}
\question[5] Complex numbers \(u\) and \(v\) are given by \(u = q + 2i\) and \(v = 1 - 2i\). Given that \(\left| \frac{u}{v} \right| = 13\), find all possible values of \(q\).
\fillwithlines{3cm}
\question[5] One solution of the equation \(2z^3 - 5z^2 + cz - 5 = 0\) is \(z = 1 - 2i\). If \(c\) is real, find the value of \(c\) and the other two solutions of the equation.
\fillwithlines{3cm}
\question[5] Find the values of \( x \) and \( y \), given that \( x \) and \( y \) are real, and

\[
\frac{1}{x + iy} - \frac{1}{1 + i} = 1 - 2i.
\]
\fillwithlines{3cm}
\question[5] If \( p = 3 - i \) and \( q = -2 + 5i \), find \( \overline{p} - 3q \), giving your solution in the form \( a + bi \).
\fillwithlines{3cm}
\question[5] Write \( \frac{3}{4 - \sqrt{5}} \) in the form \( a + b\sqrt{5} \) where \( a \) and \( b \) are rational numbers.
\fillwithlines{3cm}
\question[5] Solve the equation \( z^4 + 16p^2i = 0 \), where \( p \) is real. Write your solution in polar form, in terms of \( p \).
\fillwithlines{3cm}
\question[5] Find all possible values of \( m \) that make \( z = (\sqrt{3} + mi) / (1 + \sqrt{3}i) \) a purely real number.
\fillwithlines{3cm}
\question[5] If \( |z| = 1 \) and \( z \neq 1 \), prove that \(\frac{1+z}{1-z}\) is purely imaginary.
\fillwithlines{3cm}
\question[5] If \( u = q^2 \text{cis} \frac{3\pi}{4} \) and \( v = q^3 \text{cis} \frac{\pi}{3} \), write \( \frac{u}{v} \) in the form \( r \text{cis} \theta \).
\fillwithlines{3cm}
\question[5] If \( x \) and \( y \) are real numbers and \( (x + iy)(2 + i) = 3 - i \), find the values of \( x \) and \( y \).
\fillwithlines{3cm}
\question[5] Solve the following equation for \( x \) in terms of \( w \).

\[ 2\sqrt{x} - w\sqrt{x} = 0 \]
\fillwithlines{3cm}
\question[5] Two complex numbers are defined by \( u = 1 + pi \) and \( v = 5 + 3i \). Given that \( \arg \left( \frac{u}{v} \right) = \frac{\pi}{4} \), find the value of \( p \).
\fillwithlines{3cm}
\question[5] Prove that the quadratic equation \(x^2 + 3kx + k^2 = 7x + 3k\) will have two distinct real solutions for all real values of \(k\).
\fillwithlines{3cm}
\question[5] Given the quadratic equation \( x^2 + 3kx + k^2 = 7x + 3k \), show that it always has two real solutions for all values of \( k \).
\fillwithlines{3cm}
\question[5] If \( s = 2 + 3i \) and \( t = 3 + ki \), find the value of \( k \) if \( st = 21 - i \).
\fillwithlines{3cm}
\question[5] Find the value(s) of \( r \) such that the equation \( x^2 + 4rx + r = 0 \) has only one solution.
\fillwithlines{3cm}
\question[5] Write \(\frac{k + ki}{1 - i} + \frac{2k}{1 + i}\) in its simplest possible form.
\fillwithlines{3cm}
\question[5] Given that \(x - 2\) is a factor of \(2x^3 + qx^2 - 17x - 10\), find the value of \(q\).
\fillwithlines{3cm}
\question[5] Find all possible values of \( k \) given that \(|5 + 3ki| = 13\).
\fillwithlines{3cm}
\question[5] One of the solutions of the equation \(2z^3 - 15z^2 + bz - 30 = 0\) is \(z = 3 + i\), where \(b\) is a real number. Find the other solutions and the value of \(b\).
\fillwithlines{3cm}
\question[5] Given that \( u = p + pi \) and \( v = -q + qi \), where \( p \) and \( q \) are both positive real constants, find \( \arg \left( \frac{u}{v} \right) \).
\fillwithlines{3cm}
\question[5] Find the Cartesian equation of the locus described by \( |z + i|^2 + |z - i|^2 = 10 \). Write your solution in the form \( x^2 + y^2 = k \).
\fillwithlines{3cm}
\question[5] If \( u = 12k^3 \text{cis}(\pi) \) and \( v = 2k \text{cis} \left( \frac{\pi}{3} \right) \), write the exact value of \( \frac{u}{v} \) in polar form.
\fillwithlines{3cm}
\question[5] If \( z = 5 - i \) and \( w = -2 + 3i \), show that \( |z|^2 = 2|w|^2 \).
\fillwithlines{3cm}
\question[5] Given that \( z = a + bi \), where \( a \) and \( b \) are non-zero real numbers, show that \(\frac{z \overline{z}}{z + \overline{z}}\) is a real number.
\fillwithlines{3cm}
\question[5] Solve the equation \( z^4 = -16k^8 \), where \( k \) is a real constant. Give your solutions in polar form in terms of \( k \).
\fillwithlines{3cm}
\question[5] For complex numbers \( u \) and \( v \), prove that if \( |u + v| = |u - v| \), then \( \frac{u}{v} \) is purely imaginary.
\fillwithlines{3cm}
\question[5] Given a complex number \( z_1 = 2k^2 \text{cis} \left( \frac{\pi}{4} \right) \), express it in standard form.
\fillwithlines{3cm}
\question[5] Given that \( w = d + 5i \) and \( z = 3 - 4i \), find the value of \( d \) if the product \( wz = 38 - 9i \).
\fillwithlines{3cm}
\question[5] If \( z = 2 + 3i \), show \(\frac{26}{z}\) on the Argand diagram.
\fillwithlines{3cm}
\question[5] The polynomial \( f(x) = x^3 + 3x^2 + ax + b \) has the same remainder when divided by \( (x - 2) \) as it does when divided by \( (x + 1) \). The polynomial \( f(x) \) also has \( (x + 2) \) as a factor. Find the values of \( a \) and \( b \).
\fillwithlines{3cm}
\question[5] Show that if \( z = 1 + 3i \), then \(\arg \left( \frac{z - 1}{z - 2i} \right) = \frac{\pi}{4}\).
\fillwithlines{3cm}
\question[5] Given that the real part of \((z - 2i) / (z - 4)\) is zero and \(z \neq 4\), prove that the locus of points described by \(z\) is given by the Cartesian equation \((x - 2)^2 + (y - 1)^2 = 5\).
\fillwithlines{3cm}
\question[5] Given that \( u = 2i \) and \( w = 2 \text{cis} \left( \frac{2\pi}{3} \right) \), find \( z = \frac{u}{w} \).
\fillwithlines{3cm}
\question[5] Solve the equation \( x^2 - 12qx + 20q^2 = 0 \) for \( x \) in terms of \( q \), expressing any solutions in their simplest form.
\fillwithlines{3cm}
\question[5] Solve the equation \( z^3 = k^6 + k^6 i \), where \( k \) is a real constant.
\fillwithlines{3cm}
\question[5] If \( z \) is a complex number and \( |z + 16| = 4|z + 1| \), find the value of \( |z| \).
\fillwithlines{3cm}
\question[5] The complex number \( u = 5 + mi \) has a magnitude \( |u| = 6 \). Given that \( 0 < \arg(u) < \frac{\pi}{2} \), find the exact value of the real number \( m \).
\fillwithlines{3cm}
\question[5] Write \( \frac{18}{4 - 2\sqrt{3}} \) in the form \( a + b\sqrt{3} \), where \( a \) and \( b\) are integers.
\fillwithlines{3cm}
\question[5] One solution of \(4z^3 - 19z^2 + 128z + A = 0\) is \(z = 2 + 5i\). If \(A\) is real, find the value of \(A\) and the other two solutions of the equation.
\fillwithlines{3cm}
\question[5] Solve the following equation for \(x\) in terms of \(m\).

\[6\sqrt{2x} - 5 = 6\sqrt{2x} + m\]
\fillwithlines{3cm}
\question[5] Solve the equation \( x^2 = y^2 \) for \( x \).
\fillwithlines{3cm}
\question[5] If \( u = 3 + 2i \), \( v = 4 + 2i \), and \( w = 2 + ki \), find the value of \( k \) if \(\arg(uvw) = \frac{\pi}{4}\).
\fillwithlines{3cm}
\question[5] Find the value(s) of \( p \) for which the equation \( x - 2\sqrt{x} + p = -5 \) has only one real solution.
\fillwithlines{3cm}
\question[5] For complex numbers \( w \) and \( z \), prove that:

\[
|w + z|^2 - |w - z|^2 = 4 \operatorname{Re}(w) \operatorname{Re}(z)
\]

where \(\operatorname{Re}(w)\) is the real part of \( w \), and \(\operatorname{Re}(z)\) is the real part of \( z\).
\fillwithlines{3cm}
\question[5] Dividing \( x^3 - 3x^2 + bx + 9 \) by \( x + 2 \) gives a remainder of 3. Find the value of \( b \).
\fillwithlines{3cm}
\question[5] Find the complex number \( z \) for which \( z + 4z = 15 + 12i \).
\fillwithlines{3cm}
\question[5] One of the solutions of \( z^3 - 2z^2 + hz + 180 = 0 \) is \( z = -4 \). (\( h \) is a real number). Find the other solutions, in the form \( a \pm bi \), and the value of \( h \).
\fillwithlines{3cm}
\question[5] If \( z = 1 - \sqrt{3}i \) and \( w = \frac{4}{z} - 2 \), find \(\arg(w)\).
\fillwithlines{3cm}
\question[5] Find the Cartesian equation of the locus described by \( |z + i| = 2|z - 5i| \) in the form \((x-a)^2 + (y-b)^2 = k^2\).
\fillwithlines{3cm}
\question[5] Solve the equation \( z^2 + 6kz + 15k^2 = 0 \) in terms of the real number \( k \). Give your solution in the form \( ak \pm \sqrt{b}ki \), where \( a \) and \( b \) are rational numbers.
\fillwithlines{3cm}
\question[5] Solve the equation \( z^3 + k^6i = 0 \), where \( k \) is a real constant. Give your solution(s) in polar form in terms of \( k \).
\fillwithlines{3cm}
\question[5] Prove that there is no complex number \( z \) such that \( |z| - z = i \).
\fillwithlines{3cm}
\question[5] If \( z = a + bi \) is a non-zero complex number, and \( \frac{1}{z} + \frac{3}{\overline{z}} = 1 \), find the values of \( a \) and \( b \).
\fillwithlines{3cm}
\question[5] Write \((5 - 2\sqrt{p})^2\) in the form \(a + bp + c\sqrt{p}\) where \(a\), \(b\), and \(c\) are integers.
\fillwithlines{3cm}
\question[5] Find the value(s) of \( r \) so that the quadratic equation \( 4x^2 - 4x + 3r - 2 = 0 \) has no real roots.
\fillwithlines{3cm}
\question[5] If \( z = p + qi \) and \( w = a + bi \) and the real part of \( \frac{z}{w} \) is 0, show that \( ap = -bq \).
\fillwithlines{3cm}
\question[5] One solution of the equation \( z^3 - 8z^2 + 6z + d = 0 \) is \( z = 5 - i \). If \( d \) is real, find the value of \( d \) and the other two solutions of the equation.
\fillwithlines{3cm}
\question[5] The complex numbers \( u \) and \( v \) are given by \( u = 3 + i \) and \( v = 1 + 2i \). Determine the possible value(s) of the real constant \( k \) if \(\left| \frac{u}{v} + k \right| = \sqrt{k + 2}\).
\fillwithlines{3cm}
\question[5] If \( u = q^6 \text{ cis } \frac{5\pi}{8} \) and \( v = q^2 \text{ cis } \frac{2\pi}{5} \), write \( \frac{u}{v} \) in the form \( r \text{ cis } \theta \).
\fillwithlines{3cm}
\question[5] If \( z = 1 + ki \) and \( w = 7 - ki \), then find \( |z - w| \), giving your answer in terms of \( k \).
\fillwithlines{3cm}
\question[5] Find \( \text{Arg}(z) \) if \( \frac{13z}{z+1} = 11 - 3i \).
\fillwithlines{3cm}
\question[5] Solve the equation \( z^3 + 64m^{12} = 0 \), where \( m \) is a real constant. Write your solution(s) in polar form, in terms of \( m \).
\fillwithlines{3cm}
\question[5] The straight line with equation \( y = mx - 1 \), where \( m \) is a real constant and \( m > 0 \), is a tangent to the locus described by \( |z - 2 + i| = \sqrt{3} \). Find the Cartesian equation of the locus and the value of \( m \).
\fillwithlines{3cm}
\question[5] When the polynomial \(2x^3 + px^2 + 7x - 3\) is divided by \(x + 3\), the remainder is 30. Find the value of \(p\).
\fillwithlines{3cm}
\question[5] Differentiate \( y = \tan(x^2 + 1) \) with respect to \( x \).
\fillwithlines{3cm}
\question[5] Find the \( x \) values of any points of inflection on the graph of the function \( y = e^{(6 - x^2)} \). Show any derivatives that you need to find when solving this problem.
\fillwithlines{3cm}
\question[5] A curve is defined by the parametric equations: \( x = 5 \sin t \) and \( y = 3 \tan t \). Find the gradient of the normal to the curve at the point where \( t = \frac{\pi}{3} \). Show any derivatives that you need to find when solving this problem.
\fillwithlines{3cm}
\question[5] A closed cylindrical tank is to have a surface area of \(20 \, \text{m}^2\). Find the radius the tank needs to have so that the volume it can hold is as large as possible. Show any derivatives that you need to find when solving this problem.
\fillwithlines{3cm}
\question[5] Differentiate \( y = \sqrt[3]{\pi - x^2} \).
\fillwithlines{3cm}
\question[5] A curve has the equation \( y = (x^3 - 2x)^3 \). Find the equation of the tangent to the curve at the point where \( x = 1 \). Show any derivatives that you need to find when solving this problem.
\fillwithlines{3cm}
\question[5] For what value of \( k \) does the function \( f(x) = x - e^x - \frac{k}{x} \) have a stationary point at \( x = -1 \)? Show any derivatives that you need to find when solving this problem.
\fillwithlines{3cm}
\question[5] Differentiate \( y = \frac{\sin(2x)}{x^2} \).
\fillwithlines{3cm}
\question[5] For the function \( f(x) = x + \frac{16}{x^2 - 2} \), find the x-values of any stationary points. You must use calculus and clearly show your working, including any derivatives you need to find when solving this problem.
\fillwithlines{3cm}
\question[5] Find the value of \( x \) that gives the maximum value of the function

\[ f(x) = 50x - 30x \ln 2x \]

You do not need to prove that your value of \( x \) gives a maximum.

You must use calculus and clearly show your working, including any derivatives you need to find when solving this problem.
\fillwithlines{3cm}
\question[5] A curve is defined by the parametric equations:

\[ x = t^2 - t \quad \text{and} \quad y = t^3 - 3t \]

Find the coordinates of the point(s) on the curve for which the normal to the curve is parallel to the \( y \)-axis.

You must use calculus and clearly show your working, including any derivatives you need to find when solving this problem.
\fillwithlines{3cm}
\question[5] A spherical balloon is being inflated with helium. The balloon is being inflated in such a way that its volume is increasing at a constant rate of \(300 \, \text{cm}^3 \, \text{s}^{-1}\). The material that the balloon is made of is of limited strength, and the balloon will burst when its surface area reaches \(7500 \, \text{cm}^2\). Find the rate at which the surface area of the balloon is increasing when it reaches the bursting point. Show any derivatives that you need to find when solving this problem.
\fillwithlines{3cm}
\question[5] Solve the following mathematical problems:

1. Solve for \(x\) in the equation \(x^2 - 2x - 3 = 0\).
2. Determine the solution set for the inequality \(x^2 - 9 > 0\).
3. List the integer solutions for the equation \(x^3 + 2x^2 - 5x - 6 = 0\).
\fillwithlines{3cm}
\question[5] Differentiate \( y = 6 \tan(5x) \).
\fillwithlines{3cm}
\question[5] Find the gradient of the tangent to the function \( y = (4x - 3x^2)^3 \) at the point (1,1). You must use calculus and show any derivatives that you need to find when solving this problem.
\fillwithlines{3cm}
\question[5] Find the values of \( x \) for which the function \( f(x) = 8x - 3 + \frac{2}{x+1} \) is increasing. You must use calculus and show any derivatives that you need to find when solving this problem.
\fillwithlines{3cm}
\question[5] For what value(s) of \( x \) is the tangent to the graph of the function \( f(x) = \frac{x+4}{x(x-5)} \) parallel to the x-axis? Use calculus and show any derivatives that you need to find when solving this problem.
\fillwithlines{3cm}
\question[5] Salt harvested at the Grassmere Saltworks forms a cone as it falls from a conveyor belt. The slant of the cone forms an angle of \(30^\circ\) with the horizontal. The conveyor belt delivers the salt at a rate of 2 m\(^3\) of salt per minute. Find the rate at which the slant height is increasing when the radius of the cone is 10 m. You must use calculus and show any derivatives that you need to find when solving this problem.
\fillwithlines{3cm}
\question[5] Differentiate \( f(x) = \sqrt[3]{x - 3x^2} \).
\fillwithlines{3cm}
\question[5] Find the gradient of the normal to the curve \( y = x - \frac{16}{x} \) at the point where \( x = 4 \).

*You must use calculus and show any derivatives that you need to find when solving this problem.*
\fillwithlines{3cm}
\question[5] A street light is 5 meters above the ground, which is flat. A boy, who is 1.5 meters tall, is walking away from the point directly below the streetlight at 2 meters per second. At what rate is the length of his shadow changing when the boy is 8 meters away from the point directly under the light? You must use calculus and show any derivatives that you need to find when solving this problem.
\fillwithlines{3cm}
\question[5] The height of the tide at a particular beach today is given by the function

\[ h(t) = 0.8 \sin \left( \frac{4\pi}{25} t + \frac{\pi}{2} \right) \]

where \( h \) is the height of water, in metres, relative to the mean sea level and \( t \) is the time in hours after midnight.

At what rate was the height of the tide changing at that beach at 9.00 a.m. today?
\fillwithlines{3cm}
\question[5] A curve is defined by the parametric equations \( x = 2\cos(2t) \) and \( y = \tan^2(t) \). Find the gradient of the tangent to the curve at the point where \( t = \frac{\pi}{4} \). You must use calculus and show any derivatives that you need to find when solving this problem.
\fillwithlines{3cm}
\question[5] The tangents to the curve \( y = \frac{1}{4}(x-2)^2 \) at points \( P \) and \( Q \) are perpendicular. Given that \( Q \) is the point \( (6, 4) \), what is the \( x \)-coordinate of point \( P \)? You must use calculus and show any derivatives that you need to find when solving this problem.
\fillwithlines{3cm}
\question[5] A curve is defined by the function \( f(x) = e^{-(x-k)^2} \). Find, in terms of \( k \), the \( x \)-coordinate(s) for which \( f''(x) = 0 \). You must use calculus and show any derivatives that you need to find when solving this problem.
\fillwithlines{3cm}
\question[5] Find the gradient of the tangent to the function \( y = \sqrt{2x - 1} \) at the point (5, 3). You must use calculus and show any derivatives that you need to find when solving this problem.
\fillwithlines{3cm}
\question[5] A large spherical helium balloon is being inflated at a constant rate of 4800 cm\(^3\) s\(^{-1}\). At what rate is the radius of the balloon increasing when the volume of the balloon is 288000\(\pi\) cm\(^3\)? You must use calculus and show any derivatives that you need to find when solving this problem.
\fillwithlines{3cm}
\question[5] A cone of height \( h \) and radius \( r \) is inscribed inside a sphere of radius 6 cm. The base of the cone is \( s \) cm below the \( x \)-axis. Find the value of \( s \) which maximizes the volume of the cone. You must use calculus and show any derivatives that you need to find when solving this problem. You do not need to prove that the volume you have found is a maximum.
\fillwithlines{3cm}
\question[5] Differentiate the function \( f(x) = \sqrt[3]{3x} + 2 \).
\fillwithlines{3cm}
\question[5] Find the \( x \)-value at which a tangent to the curve \( y = 6x - e^{3x} \) is parallel to the \( x \)-axis. Use calculus and show any derivatives that you need to find when solving this problem.
\fillwithlines{3cm}
\question[5] A rectangle has one vertex at \((0,0)\) and the opposite vertex on the curve \(y = (x - 6)^2\), where \(0 < x < 6\). Find the maximum possible area of the rectangle. You must use calculus and show any derivatives that you need to find when solving this problem. You do not need to prove that the area you have found is a maximum.
\fillwithlines{3cm}
\question[5] If \( y = \frac{e^x}{\sin x} \), show that \(\frac{dy}{dx} = y(1 - \cot x)\).
\fillwithlines{3cm}
\question[5] Given the equations for the tangent of angles \(\alpha\) and \(\alpha + \theta\) in terms of distance \(d\), find the value of \(d\) that maximizes \(\tan \theta\).
\fillwithlines{3cm}
\question[5] Differentiate \( y = \sqrt{x} + \tan(2x) \).
\fillwithlines{3cm}
\question[5] Find the gradient of the tangent to the curve \( y = \frac{e^{2x}}{x+2} \) at the point where \( x = 0 \). You must use calculus and show any derivatives that you need to find when solving this problem.
\fillwithlines{3cm}
\question[5] The normal to the parabola \( y = 0.5(x - 3)^2 + 2 \) at the point (1,4) intersects the parabola again at the point P. Find the x-coordinate of point P. You must use calculus and show any derivatives that you need to find when solving this problem.
\fillwithlines{3cm}
\question[5] A curve is defined parametrically by the equations \( x = \sqrt{t+1} \) and \( y = \sin 2t \). Find the gradient of the tangent to the curve at the point when \( t = 0 \). You must use calculus and show any derivatives that you need to find when solving this problem.
\fillwithlines{3cm}
\question[5] Find the values of \( a \) and \( b \) such that the curve \( y = \frac{ax - b}{x^2 - 1} \) has a turning point at (3,1). You must use calculus and show any derivatives that you need to find when solving this problem.
\fillwithlines{3cm}
\question[5] Differentiate \( y = 2(x^2 - 4x)^5 \).
\fillwithlines{3cm}
\question[5] The percentage of seeds germinating depends on the amount of water applied to the seedbed that the seeds are sown in, and may be modeled by the function:

\[ P(w) = 96 \ln(w + 1.25) - 16w - 12 \]

where \( P \) is the percentage of seeds that germinate and \( w \) is the daily amount of water applied (litres per square metre of seedbed), with \( 0 \leq w \leq 15 \).

Find the amount of water that should be applied daily to maximize the percentage of seeds germinating. You must use calculus and show any derivatives that you need to find when solving this problem.
\fillwithlines{3cm}
\question[5] The tangent to the curve \( y = \sqrt{x} \) is drawn at the point (4, 2). Find the coordinates of the point Q where the tangent intersects the x-axis. You must use calculus and show any derivatives that you need to find when solving this problem.
\fillwithlines{3cm}
\question[5] Find the coordinates of the point \( P(x, y) \) on the curve \( y = \sqrt{x} \) that is closest to the point \( (4, 0) \). You do not need to prove that your solution is the minimum value. You must use calculus and show any derivatives that you need to find when solving this problem.
\fillwithlines{3cm}
\question[5] A rectangle is inscribed in a semi-circle of radius \( r \). Show that the maximum possible area of such a rectangle occurs when \( x = \frac{r}{\sqrt{2}} \). You must use calculus and show any derivatives that you need to find when solving this problem.
\fillwithlines{3cm}
\question[5] Differentiate \( y = x \ln(3x - 1) \).
\fillwithlines{3cm}
\question[5] Find the gradient of the curve \( y = \frac{1}{x} - \frac{1}{x^2} \) at the point \( \left( 2, \frac{1}{4} \right) \). You must use calculus and show any derivatives that you need to find when solving this problem.
\fillwithlines{3cm}
\question[5] A building has an external elevator. The elevator is rising at a constant rate of \(2 \, \text{m/s}\). Sarah is stationary, watching the elevator from a point 30 m away from the base of the elevator shaft. Let the angle of elevation of the elevator floor from Sarah's eye level be \(\theta\). Find the rate at which the angle of elevation is increasing when the elevator floor is 20 m above Sarah’s eye level. You must use calculus and show any derivatives that you need to find when solving this problem.
\fillwithlines{3cm}
\question[5] Find all the value(s) of \( k \) such that the function \( y = e^x \cos kx \) satisfies the equation \(\frac{d^2y}{dx^2} - 2 \frac{dy}{dx} + 2y = 0\) for all values of \( x \).
\fillwithlines{3cm}
\question[5] Differentiate \( y = 2x^3 + \frac{5}{(x^3 + 2)^3} \).
\fillwithlines{3cm}
\question[5] If \( f(x) = 3 \cos 3x \), show that \( 9f(x) + f''(x) = 0 \).
\fillwithlines{3cm}
\question[5] Find the gradient of the curve \( y = \ln |\sin^2 x| \) at the point where \( x = \frac{\pi}{6} \). You must use calculus and show any derivatives that you need to find when solving this problem.
\fillwithlines{3cm}
\question[5] A car is being pulled along by a rope attached to the tow-bar at the back of the car. The rope passes through a pulley, the top of which is 3 meters higher than the tow-bar. The pulley is \( x \) meters horizontally from the tow-bar. The rope is being winched in at a speed of 0.6 meters per second. The wheels of the car remain in contact with the ground. At what speed is the car moving when the length of the rope, \( L \), between the tow-bar and the pulley is 5.4 meters? You must use calculus and show any derivatives that you need to find when solving this problem.
\fillwithlines{3cm}
\question[5] A curve is defined by the parametric equations \( x = t^3 + 1 \) and \( y = t^2 + 1 \). Show that \(\frac{d^2y}{dx^2} \left( \frac{dy}{dx} \right)^4\) is a constant.
\fillwithlines{3cm}
\question[5] Differentiate \( y = 3\sqrt{x} + \csc(5x) \).
\fillwithlines{3cm}
\question[5] A particle is traveling in a straight line. The distance, in meters, traveled by the particle may be modeled by the function \( s(t) = \ln(3t^2 + 3t + 1) \) where \( t \geq 0 \) and \( t \) is time measured in seconds. Find the velocity of this particle after 2 seconds. You must use calculus and show any derivatives that you need to find when solving this problem.
\fillwithlines{3cm}
\question[5] Given that \( f'(x) = 0 \) and \( f''(x) < 0 \), what can be concluded about the function \( f(x) \)?
\fillwithlines{3cm}
\question[5] Provide an example of a function \( f(x) \) that is continuous but not differentiable.
\fillwithlines{3cm}
\question[5] Differentiate \( y = \sqrt{3x^2 - 1} \) with respect to \( x \).
\fillwithlines{3cm}
\question[5] Find the rate of change of the function \( f(t) = 5 \ln(3t - 1) \) when \( t = 4 \). Use calculus and show any derivatives that you need to find when solving this problem.
\fillwithlines{3cm}
\question[5] Find the gradient of the tangent to the curve \( y = \frac{e^{2x}}{1 + x^2} \) at the point where \( x = 2 \). Use calculus and show any derivatives that you need to find when solving this problem.
\fillwithlines{3cm}
\question[5] For what value(s) of \( x \) is the function \( y = x^3 e^x \) decreasing? You must use calculus and show any derivatives that you need to find when solving this problem.
\fillwithlines{3cm}
\question[5] The volume of a sphere is increasing. At the instant when the sphere’s radius is 0.5 m, the surface area of the sphere is increasing at a rate of 0.4 m\(^2\) s\(^{-1}\). Find the rate at which the volume of the sphere is increasing at this instant. You must use calculus and show any derivatives that you need to find when solving this problem.
\fillwithlines{3cm}
\question[5] Differentiate \( y = (2x - 5)^4 \).
\fillwithlines{3cm}
\question[5] Find the gradient of the tangent to the curve \( y = \tan 2x \) at the point on the curve where \( x = \frac{\pi}{6} \). You must use calculus and show any derivatives that you need to find when solving this problem.
\fillwithlines{3cm}
\question[5] A curve is defined parametrically by the equations \( x = \frac{1}{(5-t)^2} \) and \( y = 5t - t^2 \). Find the gradient of the tangent to the curve at the point when \( t = 2 \). You must use calculus and show any derivatives that you need to find when solving this problem.
\fillwithlines{3cm}
\question[5] The Wynyard Crossing bridge in Auckland can be raised and lowered to allow tall boats to sail through when open, and pedestrians to walk across when closed. The bridge consists of two arms, each of length 22 meters. When the bridge is rising, the angle of the bridge arm above the horizontal increases at the rate of \(0.01 \, \text{rad/s}\). Find the rate at which the height, \(BH\), is increasing when \(H\) is 15 meters above the horizontal, \(FB\). You must use calculus and show any derivatives that you need to find when solving this problem.
\fillwithlines{3cm}
\question[5] Given \( y = e^u \) and \( u = \sin 2x \), show that

\[
\frac{d^2 y}{dx^2} = \frac{d^2 y}{du^2} \left( \frac{du}{dx} \right)^2 + \frac{dy}{du} \frac{d^2 u}{dx^2}
\]

Use calculus and show any derivatives that you need to find when solving this problem.
\fillwithlines{3cm}
\question[5] Differentiate \( y = \frac{4}{\sin x} \) and find the second derivative of \( y = e^{\sin 2x} \).
\fillwithlines{3cm}
\question[5] A rectangle has one vertex at (0,0), and the opposite vertex on the curve \( y = 4 - \sqrt{x} \), where \( 0 < x < 16 \). Find the maximum possible area of the rectangle. You must use calculus and show any derivatives that you need to find when solving this problem. You do not need to prove that the area you have found is a maximum.
\fillwithlines{3cm}
\question[5] The velocity of an object is modeled by the function

\[ v = 2e^t + 8e^{-t}, \text{ for } t \geq 0 \]

where \( v \) is the velocity of the object in meters per second (m/s) and \( t \) is the time in seconds since the start of the object’s motion.

Find the time when the acceleration of the object is 0.

You must use calculus and show any derivatives that you need to find when solving this problem.
\fillwithlines{3cm}
\question[5] The graph below shows the function \( y = 2\sqrt{36 - x^2} \), and the tangent to that function at point P. The tangent intersects the x-axis at the point (8,0). Find the x-coordinate of point P. You must use calculus and show any derivatives that you need to find when solving this problem.
\fillwithlines{3cm}
\question[5] Differentiate \( y = (3x - x^2)^5 \).
\fillwithlines{3cm}
\question[5] Find the gradient of the tangent to the curve \( y = 3\sin(2x) + \cos(2x) \) at the point where \( x = \frac{\pi}{4} \). You must use calculus and show any derivatives that you need to find when solving this problem.
\fillwithlines{3cm}
\question[5] Find the value of \( x \) for which the graph of the function \( y = \frac{x}{1 + \ln x} \) has a stationary point. You must use calculus and show any derivatives that you need to find when solving this problem.
\fillwithlines{3cm}
\question[5] A curve has the equation \( y = x^2 \cos x \). Show that the equation of the tangent to the curve at the point \( \left( \pi, -\pi^2 \right) \) is \( y + 2\pi x = \pi^2 \). You must use calculus and show any derivatives that you need to find when solving this problem.
\fillwithlines{3cm}
\question[5] A cylinder of height \( h \) and radius \( r \) is inscribed inside a sphere of radius 20 cm. Find the maximum possible volume of the cylinder. You must use calculus and show any derivatives that you need to find when solving this problem. You do not need to prove that the volume you have found is a maximum.
\fillwithlines{3cm}
\question[5] Differentiate \( y = \frac{\tan x}{x^3} \).
\fillwithlines{3cm}
\question[5] The value of a car is modeled by the formula

\[ V = 17000 e^{-0.25t} + 2000 e^{-0.5t} + 500 \quad \text{for} \quad 0 \leq t \leq 20 \]

where \( V \) is the value of the car in dollars (\$), and \( t \) is the age of the car in years.

Calculate the rate at which the value of the car is changing when it is 8 years old.

You must use calculus and show any derivatives that you need to find when solving this problem.
\fillwithlines{3cm}
\question[5] Find the \( x \)-coordinates of any stationary points on the graph of the function

\[ f(x) = (2x - 3)e^{x+k} \]

You must use calculus and show any derivatives that you need to find when solving this problem.
\fillwithlines{3cm}
\question[5] A rocket is fired vertically upwards. Its height above the launch point is given by the formula \( h(t) = 4.8t^2 \) where \( h \) is the height in meters, and \( t \) is the time in seconds from firing. An observer at point A is watching the rocket. She is at the same level as the launch point of the rocket and 500 meters from the launch point. Find the rate at which the angle of elevation at A of the rocket is increasing when the rocket is 480 meters above the launch point. You must use calculus and show any derivatives that you need to find when solving this problem.
\fillwithlines{3cm}
\question[5] A curve is defined by the parametric equations \( x = \ln(t) \) and \( y = 6t^3 \) where \( t > 0 \). The point P lies on the curve, and at point P, the second derivative of \( y \) with respect to \( x \), \( \frac{d^2 y}{dx^2} \), is equal to 2. Find the exact coordinates of point P. You must use calculus and show any derivatives that you need to find when solving this problem.
\fillwithlines{3cm}
\question[5] Differentiate \( y = 3\ln(x^2 - 1) \) with respect to \( x \).
\fillwithlines{3cm}
\question[5] For what value(s) of \( x \) does the tangent to the graph of the function

\[ f(x) = 2x - 2\sqrt{x}, \, x > 0, \]

have a gradient of 1? You must use calculus and show any derivatives that you need to find when solving this problem.
\fillwithlines{3cm}
\question[5] The normal to the graph of the function \( y = \sqrt{2x+1} \) at the point \( (4,3) \) intersects the x-axis at point P. Find the x-coordinate of point P. You must use calculus and show any derivatives that you need to find when solving this problem.
\fillwithlines{3cm}
\question[5] The graph of the function \( y = \frac{1}{x-3} + x, \, x \neq 3 \), has two stationary points. Find the \( x \)-coordinates of the stationary points, and determine whether they are local maxima or local minima. You must use calculus and show any derivatives that you need to find when solving this problem.
\fillwithlines{3cm}
\question[5] A curve has the equation \( y = (3x + 2)e^{-2x} \). Prove that \(\frac{d^2 y}{dx^2} + 4 \frac{dy}{dx} + 4y = 0.\) You must use calculus and show any derivatives that you need to find when solving this problem.
\fillwithlines{3cm}
\question[5] Solve for \(x\) in the equation \(x^2 - 2x + 1 = 0\).
\fillwithlines{3cm}
\question[5] Differentiate \( y = e^{3x} \sin 2x \).
\fillwithlines{3cm}
\question[5] A curve has the equation \( y = (2x + 3)e^{x^2} \). Find the \( x \)-coordinate(s) of any stationary point(s) on the curve. You must use calculus and show any derivatives that you need to find when solving this problem.
\fillwithlines{3cm}
\question[5] A curve is defined parametrically by the equations \( x = t^2 + 3t \) and \( y = t^2 \ln(2t - 3) \), for \( t > \frac{3}{2} \). Find the gradient of the tangent to the curve at the point (10,0). You must use calculus and show any derivatives that you need to find when solving this problem.
\fillwithlines{3cm}
\question[5] A cone has a height of 3 meters and a radius of 1.5 meters. A cylinder is inscribed in the cone, with the base of the cylinder having the same center as the base of the cone. Prove that the maximum volume of the cylinder is pi cubic meters. You must use calculus and show any derivatives that you need to find when solving this problem.
\fillwithlines{3cm}
\question[5] Differentiate \( f(x) = (1 - x^2)^5 \).
\fillwithlines{3cm}
\question[5] A curve has the equation \( y = \frac{x^2}{x+1} \). Find the \( x \)-coordinate(s) of any stationary point(s) on the curve. You must use calculus and show any derivatives that you need to find when solving this problem.
\fillwithlines{3cm}
\question[5] A curve has the equation \( y = (x^2 + 3x + 2) \cos 3x \). Find the equation of the normal to the curve at the point where the curve crosses the \( y \)-axis. You must use calculus and show any derivatives that you need to find when solving this problem.
\fillwithlines{3cm}
\question[5] The volume of a spherical balloon is increasing at a constant rate of 60 cm\(^3\) per second. Find the rate of increase of the radius when the radius is 15 cm. You must use calculus and show any derivatives that you need to find when solving this problem.
\fillwithlines{3cm}
\question[5] Differentiate \( y = \frac{\cot x}{x^2 + 1} \).
\fillwithlines{3cm}
\question[5] The graph of the function \( y = 4\sqrt{x} - x + 2 \), where \( x > 0 \), has a stationary point at point Q. Find the coordinates of point Q. You must use calculus and show any derivatives that you need to find when solving this problem.
\fillwithlines{3cm}
\question[5] For what values of \( x \) is the function \( y = \frac{x}{x^2 + 4} \) increasing? Use calculus and show any derivatives that you need to find when solving this problem.
\fillwithlines{3cm}
\question[5] A curve has the equation \( y = \frac{4x + k}{4x - k} \), where \( k \) is a constant and \( x \neq \frac{k}{4} \). The point \( P \) lies on the curve and has an \( x \)-coordinate of 3. The gradient of the tangent to the curve at \( P \) is \(-\frac{8}{27}\). Find the possible value(s) of \( k \). You must use calculus and show any derivatives that you need to find when solving this problem.
\fillwithlines{3cm}
\question[5] A lamp is suspended above the center of a round table with radius \( r \). The height \( h \) of the lamp above the table is adjustable. Point \( P \) is on the edge of the table. At point \( P \), the illumination \( I \) is directly proportional to the cosine of angle \( \theta \) and inversely proportional to the square of the distance \( S \) to the lamp. That is, \( I = \frac{k \cos \theta}{S^2} \), where \( k \) is a constant. Prove that the edge of the table will have maximum illumination when \( h = \frac{r}{\sqrt{2}} \). You do not need to prove that your solution gives the maximum value. You must use calculus and show any derivatives that you need to find when solving this problem.
\fillwithlines{3cm}
\question[5] Differentiate \( y = (\ln x)^2 \) with respect to \( x \).
\fillwithlines{3cm}
\question[5] Find the \( x \)-value(s) of any stationary points on the graph of the function \( f(x) = \frac{x^2 + 1}{x} \). You must use calculus and show any derivatives that you need to find when solving this problem.
\fillwithlines{3cm}
\question[5] The graph below shows the function \( y = \sqrt{x+2} \), and the normal to the function at the point where the function intersects the \( y \)-axis. Find the coordinates of point \( P \), the \( x \)-intercept of the normal. You must use calculus and show any derivatives that you need to find when solving this problem.
\fillwithlines{3cm}
\question[5] A curve is defined parametrically by the equations \( x = 2 + 3t \) and \( y = 3t - \ln(3t - 1) \) where \( t > \frac{1}{3} \). Find the coordinates, \((x, y)\), of any point(s) on the curve where the tangent to the curve has a gradient of \(\frac{1}{2}\). You must use calculus and show any derivatives that you need to find when solving this problem.
\fillwithlines{3cm}
\question[5] If \( p \) is a positive real constant, prove that \( y = e^{px} \) does not have any points of inflection. You must use calculus and show any derivatives that you need to find when solving this problem.
\fillwithlines{3cm}
\question[5] Differentiate \( f(x) = (5x - 3) \sin(4x) \).
\fillwithlines{3cm}
\question[5] Find the gradient of the tangent to the curve \( y = \left(3x^2 - 2\right)^3 \) when \( x = 2 \). You must use calculus and show any derivatives that you need to find when solving this problem.
\fillwithlines{3cm}
\question[5] An object is traveling in a straight line. Its displacement, in meters, is given by the formula:

\[ d(t) = \frac{t^2 - 6}{2t^3} \]

where \( t > 0 \), and \( t \) is time in seconds.

Find the time(s) when the object is stationary. You must use calculus and show any derivatives that you need to find when solving this problem.
\fillwithlines{3cm}
\question[5] A rectangle has one vertex at (0,0) and the opposite vertex on the curve \( y = 6e^{-0.5x} \), where \( x > 0 \). Find the maximum possible area of the rectangle. You must use calculus and show any derivatives that you need to find when solving this problem. You do not have to prove that the area you have found is a maximum.
\fillwithlines{3cm}
\question[5] The curve with the equation \((y - 5)^2 = 16(x - 2)\) has a tangent of gradient 1 at point P. This tangent intersects the \(x\) and \(y\) axes at points R and S, respectively. Prove that the length RS is \(7\sqrt{2}\). You must use calculus and show any derivatives that you need to find when solving this problem.
\fillwithlines{3cm}
\question[5] Differentiate \( y = e^{4\sqrt{x}} \).
\fillwithlines{3cm}
\question[5] The diagram below shows the cross-section of a bowl containing water. When the height of the water level in the bowl is \( h \) cm, the volume, \( V \, \text{cm}^3 \), of water in the bowl is given by \( V = \pi \left( \frac{3}{2} h^2 + 3h \right) \). Water is poured into the bowl at a constant rate of 20 \(\text{cm}^3 \, \text{s}^{-1}\). Find the rate, in \(\text{cm} \, \text{s}^{-1}\), at which the height of the water level is increasing when the height of the water level is 3 cm. *You must use calculus and show any derivatives that you need to find when solving this problem.*
\fillwithlines{3cm}
\question[5] Find the \( x \)-value(s) of any stationary point(s) on the graph of the function \( y = 9x - 2 + \frac{3}{3x - 1} \) and determine their nature. You must use calculus and show any derivatives that you need to find when solving this problem.
\fillwithlines{3cm}
\question[5] Find the rate of change of the function \( f(t) = t^2 e^{2t} \) when \( t = 1.5 \). You must use calculus and show any derivatives that you need to find when solving this problem.
\fillwithlines{3cm}
\question[5] The graph shows the curve \( y = \frac{2}{(x+1)^3} \), along with the tangent to the curve drawn at \( x = 1 \). A second tangent to this curve is drawn which is parallel to the first tangent shown in the diagram. Find the \( x \)-coordinate of the point where this second tangent touches the curve. You must use calculus and show any derivatives that you need to find when solving this problem.
\fillwithlines{3cm}
\question[5] The diagram below shows a tangent passing through the point \( P(p, q) \), which lies on the circle with parametric equations \( x = 4 \cos \theta \) and \( y = 4 \sin \theta \). Show that the equation of the tangent line is \( px + qy = p^2 + q^2 \).
\fillwithlines{3cm}
\question[5] The graph of \( y = x(x - 2m)^2 \), where \( m > 0 \), is shown. The total shaded area between the curve and the x-axis from \( x = 0 \) to \( x = 2m \) is given by \( A = \frac{4m^4}{3} \).

A right-angled triangle is now constructed with one vertex at (0,0) and another on the curve \( y = x(x - 2m)^2 \).

Show that the maximum area of such a triangle is \( \frac{3}{8} \) of the total shaded area.

You must use calculus and show any derivatives that you need to find when solving this problem.

You do not have to prove that the area you have found is a maximum.
\fillwithlines{3cm}
\question[5] Differentiate \( f(x) = \frac{x^2}{\cos x} \).
\fillwithlines{3cm}
\question[5] Find the gradient of the tangent to the curve \( y = \cot(2x) \) at the point where \( x = \frac{\pi}{12} \). You must use calculus and show any derivatives that you need to find when solving this problem.
\fillwithlines{3cm}
\question[5] A curve is defined by the equation \( f(x) = \frac{e^x}{x^2 + 2x} \). Find the \( x \)-value(s) of any point(s) on the curve where the tangent to the curve is parallel to the \( x \)-axis. You must use calculus and show any derivatives that you need to find when solving this problem.
\fillwithlines{3cm}
\question[5] Find the \( x \)-value(s) of any points of inflection on the graph of the function \( f(x) = 3x^2 \ln(x) \). You can assume that your point(s) found are actually point(s) of inflection. You must use calculus and show any derivatives that you need to find when solving this problem.
\fillwithlines{3cm}
\question[5] Differentiate \( y = \ln(x^2 - x^4 + 1) \). You do not need to simplify your answer.
\fillwithlines{3cm}
\question[5] Find the value(s) of \( x \) where \( f'(x) = 0 \) and \( f''(x) < 0 \) are both true.
\fillwithlines{3cm}
\question[5] Find the coordinates of any stationary points on the graph of the function \( f(x) = \frac{1}{x} - \frac{2}{x^3} \), identifying their nature. You must use calculus and show any derivatives that you need to find when solving this problem.
\fillwithlines{3cm}
\question[5] A power line hangs between two poles. The equation of the curve \( y = f(x) \) that models the shape of the power line can be found by solving the differential equation:

\[ a \frac{d^2 y}{dx^2} = \sqrt{1 + \left( \frac{dy}{dx} \right)^2} \]

Use differentiation to verify that the function \( y = \frac{a}{2} \left( e^{\frac{x}{a}} + e^{-\frac{x}{a}} \right) \) satisfies the above differential equation, where \( a \) is a positive constant.
\fillwithlines{3cm}
\question[5] Evaluate the integral \(\int (\pi - e^{2x}) \, dx\).
\fillwithlines{3cm}
\question[5] An object is moving in a straight line. The velocity of the object is given by \(v = 6 - \frac{5}{t+1}\), where \(t\) is the time measured in seconds from when the object started moving, and \(v\) is the velocity measured in meters per second. How far does the object travel during its 4th second of motion? Provide the result of any integration needed to solve this problem.
\fillwithlines{3cm}
\question[5] A property owner assumes that the rate of increase of the value of his property at any time is proportional to the value, \( V \), of the property at that time. (i) Write the differential equation that expresses this statement.
\fillwithlines{3cm}
\question[5] The property was valued at $365,000 in May 2012, and at $382,000 in November 2013. Solve the differential equation for exponential growth to find the price the owner would have paid in May 2007 when he bought his house, assuming the model is accurate.
\fillwithlines{3cm}
\question[5] The energy required to pump water out of a tank with a circular cross-section and height \( H \) is given by:

\[ 
E = \int_0^H k(H-h)A(h) \, dh 
\]

where  
\( k \) is a constant,  
\( h \) is the height of the water in the tank at any instant,  
\( r \) is the radius of the water surface at this instant,  
\( A(h) \) is the area of the surface of water at this instant.

A cylindrical tank and a conical tank are both full of water. Both have height \( H \), and the radius at the top of both tanks is \( R \).

Show that the energy required to empty the conical tank is one sixth the energy required to empty the cylindrical tank. Provide the results of any integration needed to solve this problem.
\fillwithlines{3cm}
\question[5] Evaluate the integral \(\int_{1}^{k} 3\sqrt{x} \, dx\), expressing your answer in terms of \(k\).
\fillwithlines{3cm}
\question[5] Use integration to find the area enclosed between the graphs of the functions \(3y = x^2\) and \(y = 2x\). You must use calculus and give the result of any integration needed to solve this problem.
\fillwithlines{3cm}
\question[5] A large tank initially contains 20 liters of diesel. The tank is being filled at a rate of \(400/(t+2)^2\) liters per minute, where t is the time in minutes since the filling started. How much diesel will be in the tank 6 minutes after filling started? Give the result of any integration needed to solve this problem.
\fillwithlines{3cm}
\question[5] A curve \( y = f(x) \), which passes through the origin, is shown on the graph below. Its gradient at any point is given by the equation \( f'(x) = 1 - \frac{1}{3}x^2 \). The line on the graph is the tangent to the curve at \( x = -1 \). Find the shaded area using calculus and provide the result of any integration needed to solve this problem.
\fillwithlines{3cm}
\question[5] Find the area enclosed between the graph of \( y = \sin(2x) \), the x-axis, and the lines \( x = \frac{\pi}{6} \) and \( x = \frac{\pi}{3} \). Give the result of any integration needed to solve this problem.
\fillwithlines{3cm}
\question[5] Find the value of \( m \) such that \(\int_{m}^{2m} \frac{2x+5}{x^2+5x} \, dx = \ln 3\). Give the result of any integration needed to solve this problem.
\fillwithlines{3cm}
\question[5] The motion of an object is described by the equation \( \frac{dv}{dt} = -kv^2 \), where \( v \) is the velocity of the object in meters per second, \( t \) is the time in seconds, and \( k \) is a constant. The initial velocity of the object is \( u \) meters per second. Show that, after one second, the velocity of the object is \( v = \frac{u}{ku + 1} \). Provide the result of any integration needed to solve this problem.
\fillwithlines{3cm}
\question[5] Ben leaves his cup of coffee on the table to cool. The room’s temperature remains constant at 18°C. The rate at which the temperature of the coffee changes at any instant is proportional to the difference between the temperature of the coffee and the room temperature at that instant. (i) Write the differential equation that expresses this statement.
\fillwithlines{3cm}
\question[5] Find \(\int \left( \frac{2}{x} - \frac{3}{x^2} \right) \, dx\).
\fillwithlines{3cm}
\question[5] Find the area enclosed between the graph of \(y = 3 \sec^2 x\), the x-axis, and the lines \(x = \pi/6\) and \(x = \pi/4\). Provide the result of any integration needed to solve this problem.
\fillwithlines{3cm}
\question[5] The velocity of an object is given by \(v(t) = 5(4 - 3e^{-0.2t})\), where \(t\) is the time in seconds since the timing started and \(v\) is the velocity in meters per second. What distance did the object move in the first 10 seconds of its timed motion? Provide the result of any integration needed to solve this problem.
\fillwithlines{3cm}
\question[5] A tank holds 2500 liters of water. The tank develops a small hole in its base, and water leaks out at a rate proportional to the square root of the volume of water remaining in the tank at any instant. Two days after the leak started, 475 liters of water have leaked out of the tank. How long will it take the tank to empty completely? Give the result of any integration needed to solve this problem.
\fillwithlines{3cm}
\question[5] Find the integral \(\int (\sec x \tan x - \sin 2x) \, dx\).
\fillwithlines{3cm}
\question[5] Solve the differential equation \(\frac{d^2 y}{dx^2} = 6x^2 - 6x\), given that when \(x = 2\), \(y = 10\), and \(\frac{dy}{dx} = 8\).
\fillwithlines{3cm}
\question[5] The graph below shows the function \( y = \sin \left( \frac{x}{2} \right) \) and the lines \( x = k \) and \( x = \pi \). Find the value of \( k \) so that the areas A and B are equal. You must use calculus and give the results of any integration needed to solve this problem.
\fillwithlines{3cm}
\question[5] Given the differential equation \(\frac{dy}{dx} = \frac{3\sqrt{x}}{2y}\) and the initial condition \(y = 5\) when \(x = 4\), find the value of \(y\) when \(x = 9\).
\fillwithlines{3cm}
\question[5] The center of mass of an object is called the centroid. For a uniformly thin object, the centroid is at (x̄, ȳ) where

x̄ = \((1/A) ∫[a to b] x f(x) dx \)and \(ȳ = (1/A) ∫[a to b] (f(x)/2)^2 dx\)

A = area of the object

a and b are the lower and upper limits of x, respectively.

The shape shown shaded in the diagram below is bounded by part of the curve y = sqrt(x) + 1 and the lines x = 0, x = 4, and y = 0.

Find the coordinates (x̄, ȳ) of the centroid of the shape.

You must use calculus and give the results of any integration needed to solve this problem.
\fillwithlines{3cm}
\question[5] Use the values given below to find an approximation to \(\int_{2}^{5} f(x) \, dx\), using Simpson’s Rule.

\[
\begin{array}{c|ccccccc}
x & 2 & 2.5 & 3 & 3.5 & 4 & 4.5 & 5 \\
\hline
f(x) & 0.8 & 1.12 & 2.02 & 2.17 & 2.28 & 1.56 & 1.2 \\
\end{array}
\]
\fillwithlines{3cm}
\question[5] An oven tray is taken from a hot oven and placed in a room that has a constant temperature of 20°C. The rate at which the temperature of the oven tray changes at any instant is proportional to the difference between the temperature of the oven tray and the room temperature at that instant. Write a differential equation that models this situation.
\fillwithlines{3cm}
\question[5] Find the integral of \(\sqrt{x} + 6 \cos(2x)\) with respect to \(x\).
\fillwithlines{3cm}
\question[5] Given that \(\frac{dy}{dx} = \frac{e^{2x}}{4y}\) and \(y = 4\) when \(x = 0\), find the value of \(y\) when \(x = 2\).
\fillwithlines{3cm}
\question[5] Use integration to find the area enclosed between the curve \( y = \frac{5x - 3}{x + 3} \) and the lines \( y = 0 \), \( x = 2 \), and \( x = 5 \). Show your working. You must use calculus and give the results of any integration needed to solve this problem.
\fillwithlines{3cm}
\question[5] The graph below shows the function \( y = \cos x \), between \( x = 0 \) and \( x = \frac{\pi}{2} \), rotated around the \( x \)-axis. Find the volume created by this rotation. *You must use calculus and give the results of any integration needed to solve this problem.*
\fillwithlines{3cm}
\question[5] An object originally moving at a constant velocity suddenly starts to accelerate. From the start of the object’s acceleration, the motion of the object can be modeled by the differential equation

\[ \frac{dv}{dt} = \frac{50t^2 - 80\sqrt{t}}{5\sqrt{t}} \]

for \( 0 \leq t \leq 20 \), where \( v \) is the velocity of the object in m/s and \( t \) is the time in seconds after the object starts to accelerate.

If the original velocity of the object was 6 m/s, find the velocity of the object when \( t = 4 \).

You must use calculus and give the results of any integration needed to solve this problem.
\fillwithlines{3cm}
\question[5] In the town of Clarkeville, the rate at which the population, \( P \), of the town changes at any instant is proportional to the population of the town at that instant. Write a differential equation that models this situation.
\fillwithlines{3cm}
\question[5] At the start of 2000, the population of the town was 12,000. At the start of 2010, the population of the town was 16,000. Solve the differential equation to find the population the town will have at the start of 2025. You must use calculus and give the results of any integration needed to solve this problem.
\fillwithlines{3cm}
\question[5] Evaluate the integral \(\int \frac{2x^4 - x^2}{x^3} \, dx\).
\fillwithlines{3cm}
\question[5] Evaluate the integral \(\int \sec(3x) \tan(3x) \, dx\).
\fillwithlines{3cm}
\question[5] If dy/dx = (cos x)/(3y) and y = 1 when x = π/6, find the value of y when x = 7π/6. You must use calculus and give the results of any integration needed to solve this problem.
\fillwithlines{3cm}
\question[5] Use integration to find the area enclosed between the curve \( y = e^{2x} - \frac{1}{e^{3x}} \) and the lines \( y = 0 \), \( x = 0 \), and \( x = 1.2 \). You must use calculus and give the results of any integration needed to solve this problem.
\fillwithlines{3cm}
\question[5] Mr. Newton has a container of oil and places it in the garage. Unfortunately, he puts the container on top of a sharp nail, and it begins to leak. The rate of decrease of the volume of oil in the container is given by the differential equation \(\frac{dV}{dt} = -kVt\), where \(V\) is the volume of oil remaining in the container \(t\) hours after the container was put in the garage. The volume of oil in the container when it was placed in the garage was 3000 mL. After 20 hours, the volume of oil in the container was 2400 mL. How much, if any, of the oil will remain in the container 96 hours after it was placed in the garage? You must use calculus and give the results of any integration needed to solve this problem.
\fillwithlines{3cm}
\question[5] Find \( \int (5x^2 - 1)^2 \, dx \).
\fillwithlines{3cm}
\question[5] The acceleration of an object is given by \( a(t) = 0.2t + 0.3\sqrt{t} \) for \( 0 \leq t \leq 10 \). Where \( a \) is the acceleration of the object in m/s\(^2\) and \( t \) is the time in seconds from when the object started to move. The object was moving with a velocity of 5 m/s when \( t = 4 \). How far was the object from its starting point after 9 seconds? *You must use calculus and give the results of any integration needed to solve this problem.*
\fillwithlines{3cm}
\question[5] Find the value of the constant \( m \) such that

\[
\int_{m}^{2m} (2x - m)^2 \, dx = 117.
\]

You must use calculus and give the results of any integration needed to solve this problem.
\fillwithlines{3cm}
\question[5] Find \(\int 4 \sec^2(2x) \, dx\).
\fillwithlines{3cm}
\question[5] Use integration to find the area enclosed between the curve \( y = \frac{x^2 + \sqrt{x}}{x} \) and the lines \( y = 0 \), \( x = 1 \), and \( x = 4 \). You must use calculus and show the results of any integration needed to solve the problem.
\fillwithlines{3cm}
\question[5] An object's acceleration is modeled by the function \( a(t) = 1.2\sqrt{t} \) where \( a \) is the acceleration of the object in meters per second squared (m/s\(^2\)), and \( t \) is the time in seconds since the start of the object's motion. If the object had a velocity of 7 meters per second (m/s) after 4 seconds, how far did it travel in the first 9 seconds of motion? You must use calculus and show the results of any integration needed to solve the problem.
\fillwithlines{3cm}
\question[5] Find the value of \( k \) if

\[ \int_{0}^{k} 3e^{2x} \, dx = 4. \]

You must use calculus and show the results of any integration needed to solve the problem.
\fillwithlines{3cm}
\question[5] The mean value of a function \( y = f(x) \) from \( x = a \) to \( x = b \) is given by

\[
\text{Mean value} = \frac{1}{b-a} \int_{a}^{b} f(x) \, dx
\]

Find the mean value of \( y = \sin^2 x \) between \( x = 0 \) and \( x = \pi \).

You must use calculus and show the results of any integration needed to solve the problem.
\fillwithlines{3cm}
\question[5] Find \(\int \frac{6}{2x-1} \, dx\).
\fillwithlines{3cm}
\question[5] Find \(\int (2x-5)^4 \, dx\).
\fillwithlines{3cm}
\question[5] Part of the graph of \( y = \sin 3x \cos 2x \) is shown below. Find the area enclosed between the curve \( y = \sin 3x \cos 2x \) and the lines \( y = 0 \), \( x = 0 \), and \( x = \frac{\pi}{4} \). You must use calculus and show the results of any integration needed to solve the problem.
\fillwithlines{3cm}
\question[5] Find \(\int \left( \frac{9}{x^4} + 8e^{4x} \right) \, dx\).
\fillwithlines{3cm}
\question[5] Julia’s friend Sarah believes that the equation of the curved border of the paved courtyard can be modeled by the function \( y = \frac{15x - 15}{x + 2} \). Use integration to find the area of the courtyard from \( x = 1 \) to \( x = 3 \). *You must use calculus and show the results of any integration needed to solve the problem.*
\fillwithlines{3cm}
\question[5] Solve the differential equation \(\frac{dy}{dx} = \frac{y}{\sqrt{x}}\), given that when \(x = 4\), \(y = 1\). You must use calculus and show the results of any integration needed to solve the problem.
\fillwithlines{3cm}
\question[5] Find the integral \(\int \left( 6x - \frac{8}{x^3} \right) \, dx\).
\fillwithlines{3cm}
\question[5] Solve the differential equation \(dy/dx = e^(2x) + 1/x\), given that when x = 1, y = 2.
\fillwithlines{3cm}
\question[5] Find \(\int_{6}^{8} \frac{2x - 7}{x - 5} \, dx\). You must use calculus and show the results of any integration needed to solve the problem.
\fillwithlines{3cm}
\question[5] The diagram below shows the graph of the function \( f(x) = \frac{1}{2} (e^x - 1) \).

The point \( Q(k, k) \) lies on the curve.  
The shaded region in the diagram is bounded by the curve, the x-axis, and the line \( x = k \).

Show that the shaded region has an area of \( \frac{1}{2} k \).

You must use calculus and show the results of any integration needed to solve the problem.
\fillwithlines{3cm}
\question[5] Find \(\int (\sec^2 x + \sec 2x \tan 2x) \, dx\).
\fillwithlines{3cm}
\question[5] Find the value of \( k \), given that \(\int_1^k \sqrt{x} \, dx = \frac{52}{3}\). You must use calculus and show the results of any integration needed to solve the problem.
\fillwithlines{3cm}
\question[5] The diagram below shows the graphs of the functions \( y = \cos^2 x \) and \( y = \sin^2 x \).

Find the value of \( k \) such that

\[
\int_{0}^{k} (\cos^2 x - \sin^2 x) \, dx = \frac{1}{2}.
\]

You must use calculus and show the results of any integration needed to solve the problem.
\fillwithlines{3cm}
\question[5] An object’s acceleration can be modeled by the equation \( a(t) = \frac{2}{\sqrt{t+1}} \), where \( t \geq 0 \). Here, \( a \) is the acceleration of the object in m/s\(^2\) and \( t \) is the time in seconds from the start of timing. The object has a velocity of 9 m/s when \( t = 3 \). How far did the object travel in the first 8 seconds of its timed motion? You must use calculus and show the results of any integration needed to solve the problem.
\fillwithlines{3cm}
\question[5] The mass, \( m \) grams, of a burning candle \( t \) hours after it was first lit can be modeled by the differential equation

\[ \frac{dm}{dt} = -k(m - 10) \]

where \( k > 0 \) and \( m \geq 10 \). The initial mass of the candle was 140 grams. Three hours later, the mass of the candle had halved.

Find the length of time it will take for the mass of the candle to reduce to 50 grams.

You must use calculus and show the results of any integration needed to solve the problem.
\fillwithlines{3cm}
\question[5] Find the integral \(\int \left( (4x)^2 + 4x + \frac{4}{x} \right) \, dx\).
\fillwithlines{3cm}
\question[5] Use the values given in the table below to find an approximation to the integral from 0 to 3 of f(x) dx, using Simpson’s Rule.

\[
\begin{array}{c|ccccccc}
x & 0 & 0.5 & 1 & 1.5 & 2 & 2.5 & 3 \\
\hline
f(x) & 0.3 & 0.75 & 1.1 & 1.35 & 1.6 & 1.15 & 0.5 \\
\end{array}
\]
\fillwithlines{3cm}
\question[5] Find \(\int \left( 2 + \frac{2}{\sqrt{x}} \right) \, dx\).
\fillwithlines{3cm}
\question[5] Use the values given in the table below to find an approximation to \(\int_{2}^{5} f(x) \, dx\) using the Trapezium Rule.
\fillwithlines{3cm}
\question[5] Find the integral of \(\cos(4x) \cdot \cos(2x)\) from 0 to \(\pi/12\).
\fillwithlines{3cm}
\question[5] The rate of change of quantity \( N \) at any instant is given by the differential equation:

\[
\frac{dN}{dt} = kN
\]

If \( N \) has positive values \( N_1 \) and \( N_2 \) at times \( t_1 \) and \( 2t_1 \) respectively, prove that

\[
k = \frac{1}{t_1} \ln \left( \frac{N_2}{N_1} \right)
\]

You must use calculus and show the results of any integration needed to solve the problem.
\fillwithlines{3cm}
\question[5] Find \( k \) such that \(\int_{3}^{k} \frac{8}{2x-5} \, dx = 10\). You must use calculus and show the results of any integration needed to solve the problem.
\fillwithlines{3cm}
\question[5] The diagram below shows the graph of the function \( y = \cos^2 x \). Find the area of the shaded region from \( x = 0 \) to \( x = \pi \). You must use calculus and show the results of any integration needed to solve the problem.
\fillwithlines{3cm}
\question[5] Find the integral \(\int 24(2x-1)^3 \, dx\).
\fillwithlines{3cm}
\question[5] Solve the differential equation \( \frac{dy}{dx} = 4 \sec^2(2x) \), given that when \( x = \frac{\pi}{8} \), \( y = 5 \).
\fillwithlines{3cm}
\question[5] Given that \(dy/dx = (4x)/(4x^2 - 3) + sqrt(x)\) and y(1) = 2, find y(4).
\fillwithlines{3cm}
\question[5] Find \(\int \left( x + 2 + \frac{3}{x} \right) \, dx\).
\fillwithlines{3cm}
\question[5] For \(t \geq 0\), the velocity of an object is given by \(v(t) = 0.6 \sqrt{t}\), where \(v\) is the velocity of the object in cm/s and \(t\) is the time in seconds from the start of the object’s motion. The object has a displacement of 5 cm at \(t = 0\). What will be the displacement of the object after 16 seconds?
\fillwithlines{3cm}
\question[5] Find \(\int_{4}^{8} \frac{5x - 11}{x - 3} \, dx\). You must use calculus and show the results of any integration needed to solve the problem.
\fillwithlines{3cm}
\question[5] The graph below shows the curve \( y = x + \frac{3}{x} \) and the line \( y = 4 \). Find the shaded area between the curve and the line. *You must use calculus and show the results of any integration needed to solve the problem.*
\fillwithlines{3cm}
\question[5] Find \(\int \left( \pi - \frac{2}{x^2} \right) \, dx\).
\fillwithlines{3cm}
\question[5] Use the values given in the table below to find an approximation to \(\int_0^3 f(x) \, dx\) using Simpson’s Rule.

\[
\begin{array}{c|ccccccc}
x & 0 & 0.5 & 1 & 1.5 & 2 & 2.5 & 3 \\
\hline
f(x) & 1.1 & 1.8 & 2.1 & 2.4 & 2.7 & 1.8 & 1.3 \\
\end{array}
\]
\fillwithlines{3cm}
\question[5] If \( \frac{dy}{dx} = \sqrt{y} \cdot \cos 4x \) and \( y = 1 \) when \( x = \frac{\pi}{8} \), find the value of \( y \) when \( x = \frac{\pi}{4} \). You must use calculus and show the results of any integration needed to solve the problem.
\fillwithlines{3cm}
\question[5] The graph below shows the curve \( y = x + 2\sqrt{x - 3} \). Find the shaded area between \( x = 3 \) and \( x = 4 \). *You must use calculus and give the results of any integration needed to solve this problem.*
\fillwithlines{3cm}
\question[5] Find \(\int \sec 2x \tan 2x \, dx\).
\fillwithlines{3cm}
\question[5] If dy/dx = cos(2x) and y = 1 when x = π/12, find the value of y when x = π/4. *You must use calculus and give the results of any integration needed to solve this problem.*
\fillwithlines{3cm}
\question[5] An object originally moving at a constant velocity suddenly starts to accelerate. From the start of the object’s acceleration, the motion of the object can be modeled by the differential equation

\[ \frac{dv}{dt} = t + e^{0.2t} \quad \text{for } 0 \leq t \leq 15 \]

where \( v \) is the velocity of the object in m/s and \( t \) is the time in seconds after the object starts to accelerate.

When \( t = 0 \), the velocity of the object was 8 m/s.

Find the velocity of the object when \( t = 10 \).

You must use calculus and give the results of any integration needed to solve this problem.
\fillwithlines{3cm}
\question[5] In radioactive decay, the rate at which the radioactive substance decays at any instant is proportional to the number of radioactive atoms present at that instant. This can be modeled by the differential equation dN/dt = -kN, where N is the number of radioactive atoms present and t is the time in days. A quantity of manganese-52 is produced. Manganese-52 is a radioactive isotope of manganese with a half-life of 5.6 days. How long would it take for 95% of the manganese-52 to decay? You must use calculus and give the results of any integration needed to solve this problem.
\fillwithlines{3cm}
\question[5] The graph below shows the curves \( y = \cos x \) and \( y = \cos^3 x \) for \( 0 \leq x \leq \frac{\pi}{2} \). Find the shaded area. You must use calculus and give the results of any integration needed to solve this problem.
\fillwithlines{3cm}
\question[5] Find \(\int \left( \frac{x}{3} + \frac{3}{x} \right) \, dx\).
\fillwithlines{3cm}
\question[5] The gradient function of a curve is \( \frac{dy}{dx} = \frac{8}{x^3} \). (i) Find the equation of the curve if it passes through the point (1, 3). You must use calculus and show the results of any integration needed to solve the problem.
\fillwithlines{3cm}
\question[5] (ii) Find the area enclosed by the curve \( y = -4x^{-2} + 7 \), the \(x\)-axis, and the lines \(x = 1\) and \(x = 2\). You must use calculus and show the results of any integration needed to solve the problem.
\fillwithlines{3cm}
\question[5] An object's motion can be modeled by the differential equation \( a(t) = 2 - \sin(2t) \), where \( t \geq 0 \). \( a \) is the acceleration of the object, in meters per second squared (m/s\(^2\)), and \( t \) is time in seconds. At \( t = 0 \), the object has a velocity of 1 meter per second (m/s) and a displacement of 3 meters. What is the displacement of the object at time \( t = 5 \)? You must use calculus and show the results of any integration needed to solve the problem.
\fillwithlines{3cm}
\question[5] A water tank developed a leak. 6 hours after the tank started to leak, the volume of water in the tank was 400 liters. 10 hours after the tank started to leak, the volume of water in the tank was 256 liters. The rate at which the water leaks out of the tank at any instant is proportional to the square root of the volume of the water in the tank at that instant. How much water was in the tank at the instant it started to leak? You must use calculus and show the results of any integration needed to solve the problem.
\fillwithlines{3cm}
\question[5] Find \(\int (e^{4x} + 4\sqrt{x}) \, dx\).
\fillwithlines{3cm}
\question[5] If \(\int_1^5 h(x) \, dx = 6\), what is the value of \(\int_1^5 (h(x) + 2) \, dx\)?
\fillwithlines{3cm}
\question[5] Find the integral \(\int_0^{\frac{\pi}{8}} \sin(6x) \sin(2x) \, dx\).
\fillwithlines{3cm}
\question[5] Use the values given in the table below to find an approximation to the integral from 1 to 2.5 of f(x) dx using the Trapezium Rule.

\[
\begin{array}{c|ccccccc}
x & 1 & 1.25 & 1.5 & 1.75 & 2 & 2.25 & 2.5 \\
\hline
f(x) & 0.8 & 1.1 & 1.5 & 1.9 & 2.2 & 2.1 & 2.4 \\
\end{array}
\]
\fillwithlines{3cm}
\question[5] Consider the differential equation \(dy/dx = sec^2(2x) / y\). Given that y = 2 when x = 3π/8, find the value(s) of y when x = π. You must use calculus and show the results of any integration needed to solve the problem.
\fillwithlines{3cm}
\question[5] The graph below shows the functions \( y = (ke^x)^2 \) and \( y = k \), where \( k \) is a constant greater than 1. Show that the shaded area is \(\frac{k}{2} \left( k - 1 + \ln \frac{1}{k} \right)\). You must use calculus and show the results of any integration needed to solve the problem. Clearly show each step of your working.
\fillwithlines{3cm}
\question[5] Find \(\int \left( \frac{4}{x} - \sec^2 x \right) \, dx\).
\fillwithlines{3cm}
\question[5] Find \(\int_{0}^{\frac{\pi}{4}} \sin^2(2x) \, dx\). You must use calculus and show the results of any integration needed to solve the problem.
\fillwithlines{3cm}
\question[5] The graph below shows the functions \( y = (e^x)^2 \) and \( y = 3e^x + 10 \). Find the exact value of the shaded area. You must use calculus and show the results of any integration needed to solve the problem.
\fillwithlines{3cm}
\question[5] Find \(\int (e^{3x} - \sqrt{x}) \, dx\).
\fillwithlines{3cm}
\question[5] Find the value of \( k \), given that \(\int_{1}^{k} \frac{2}{\sqrt{x}} \, dx = 8\). You must use calculus and show the results of any integration needed to solve the problem.
\fillwithlines{3cm}
\question[5] Consider the differential equation \( \frac{dy}{dx} = \frac{1}{3y^2(x-1)} \), where \( x > 1 \). Given that \( y = -1 \) when \( x = 2 \), find the value(s) of \( x \) which give a \( y \) value of 1. You must use calculus and show the results of any integration needed to solve the problem.
\fillwithlines{3cm}
\question[5] An object’s acceleration can be modeled by the equation \( a(t) = 0.9e^{0.3t} \), where \( a \) is the acceleration of the object in m/s\(^2\), and \( t \) is the time in seconds from the start of timing. The object had a velocity of 10 m/s after 2 seconds. How far did the object travel during the 5th second of its motion? Use calculus and show the results of any integration needed to solve the problem.
\fillwithlines{3cm}
\question[5] A cylindrical tank of height 150 cm is originally full of oil. The tank starts to leak out of a hole in its side. The height \( h \), in cm, of the oil left in the tank after it has been leaking for \( t \) minutes can be modeled by the differential equation

\[ 
\frac{dh}{dt} = -\frac{1}{4} \sqrt{(h-6)^3}. 
\]

Find how long it takes for the oil in the tank to be 15 cm above the bottom of the tank. You must use calculus and show the results of any integration needed to solve the problem.
\fillwithlines{3cm}
\question[5] Use the values given in the table below to find an approximation to \(\int_0^2 f(x) \, dx\) using the Trapezium Rule.

\[
\begin{array}{c|cccccc}
x & 0 & 0.4 & 0.8 & 1.2 & 1.6 & 2.0 \\
\hline
f(x) & 3.6 & 4.2 & 4.8 & 5.4 & 4.5 & 3.2 \\
\end{array}
\]
\fillwithlines{3cm}
\question[5] Find \(\int_5^8 \frac{4x-5}{x-3} \, dx\). *You must use calculus and show the results of any integration needed to solve the problem.*
\fillwithlines{3cm}
\question[5] The graph below shows part of the curve \( y = x + \cos x \) and the line \( y = x \). Find the shaded area between the curve and the line. You must use calculus and show the results of any integration needed to solve the problem.
\fillwithlines{3cm}
\question[5] The graph below shows part of the curve given by the equation \( y = \frac{2}{x} \).

Points P and Q lie on the curve with \( x \)-coordinates \( k \) and \( 3k \) respectively, where \( k > 0 \).

Point R is such that PR is parallel to the \( x \)-axis and QR is parallel to the \( y \)-axis.

The shaded area can be written in the form \( a + b \ln c \), where \( a, b, \) and \( c \) are integers.

Find the values of \( a, b, \) and \( c \).

You must use calculus and show the results of any integration needed to solve the problem.
\fillwithlines{3cm}
\question[5] Find \(\int \left( 3x + 2 + \frac{1}{3x + 2} \right) \, dx\).
\fillwithlines{3cm}
\question[5] An object’s velocity can be modeled by the equation \(v(t) = \sec^2 t\), where \(v\) is the velocity of the object in km/hr, and \(t\) is the time in hours from the start of timing. Initially, the object was 3 km from a point P. Find the distance of this object from the point P after \(\pi/4\) hours. You must use calculus and show the results of any integration needed to solve the problem.
\fillwithlines{3cm}
\question[5] The graph below shows the functions \( y = \sqrt{x} \) and \( y = \frac{x^2}{8} \). Find the shaded area between the curves. You must use calculus and show the results of any integration needed to solve the problem.
\fillwithlines{3cm}
\question[5] Consider the differential equation \( \frac{dy}{dx} = y(2x - 3x^2) \). Given that \( y = 1 \) when \( x = 2 \), find the value(s) of \( y \) when \( x = 1 \). You must use calculus and show the results of any integration needed to solve the problem.
\fillwithlines{3cm}
\question[5] Evaluate the integral of \(4e^{2x-1}\) with respect to \(x\).
\fillwithlines{3cm}
\question[5] Evaluate the integral \(\int 4e^{2x-1} \, dx\).
\fillwithlines{3cm}
\question[5] Solve the differential equation \(\frac{dy}{dx} = (4x+1)^{-1/2}\), where \(x \geq 0\), given that when \(x = 6\), \(y = 7.5\). You must use calculus and show the results of any integration needed to solve the problem.
\fillwithlines{3cm}
\question[5] Find the value of \( k \), given that

\[
\int_{2}^{k} \frac{6x - 3}{2x - 3} \, dx = 3k.
\]
\fillwithlines{3cm}
\question[5] A cake factory has a container of liquid chocolate that is used in the manufacture of chocolate cakes. The liquid chocolate is pumped out of the container so that the rate of change of the volume of liquid chocolate remaining in the container is proportional to the square of the volume of liquid chocolate remaining. After one hour of use on a particular day, the volume of chocolate remaining is \( p \) liters, where \( p \) is a positive constant. After a further one hour, there are only \( \frac{4}{5} p \) liters of chocolate remaining in the container. Write a differential equation that models this situation, and solve it to calculate how much liquid chocolate was in the container at the start of the day, giving your answer in terms of \( p \).
\fillwithlines{3cm}
\question[5] Evaluate the integral of the function \( f(x) = 1 - \frac{3}{x^2} \).
\fillwithlines{3cm}
\question[5] Find \(\int \left( \frac{\sqrt{x} - 3}{\sqrt{x}} \right) \, dx\).
\fillwithlines{3cm}

\end{questions}
\end{document}
